\documentclass[xetex,mathserif,serif]{beamer}

\usepackage[export]{adjustbox}
\usepackage{listings}
\usepackage{color}

\usetheme{Darmstadt}

\definecolor{UBCblue}{rgb}{0.29412, 0.2745, 0.53334} % UBC Blue (primary)

\usecolortheme[named=UBCblue]{structure}

\usepackage{fontspec}
\usepackage{multicol}
\usepackage{csquotes}

\title {Ontologije}
\subtitle{Seminarski rad u okviru kursa\\Bioinformatika}
\author[Crnobrnja, Mićić] % (optional, for multiple authors)
{Marko Crnobrnja i Marko Mićić}
\institute% (optional)
{
  Matematički fakultet\\
  Univerzitet u beogradu
}
\date % (optional)



\begin{document}
  \frame{\titlepage}
  
  \begin{frame}
    \frametitle{Uvod}
		\begin{center}		
		\begin{itemize}
			\item Velika količina podataka
			\item Potreban je način za usaglašeno upravljanje ovim znanjem
		\end{itemize}
			\begin{displayquote}
				Specifikacija reprezentativnog vokabulara za deljeni domen diskursa-definicije klasa, relacija, funkcija i drugih obejkata-zove se ontologija.
			\end{displayquote}			
		\begin{itemize}
			\item Definišu klase, objekte i relacije među njima
			\item Glavni ciljevi
				\begin{itemize}
			 		\item Formalizacija strukture specifičnih informacija o domenu
			 		\item Razdvajanje znanja od strukturi podataka od samih podataka
			 		\item Omogućivanje ponovnog korišćenja i deljenje strukture i znanja
			 	\end{itemize}
		\end{itemize}
	\end{center}
  \end{frame}  
  \begin{frame}
    \frametitle{Materijali}
		\begin{center}		
		\begin{itemize}
			\item U poslednjoj deceniji predložen, definisan i objavljen veliki broj ontologija iz biomedicine
				\begin{itemize}
					\item Chemical Entities of Biological Interest (ChEBI)
					\item Gene Ontology (GO)
					\item Biological Pathways Exchange (BioPAX)
				\end{itemize}
			\item Biomedical Ontologies Foundry
			\item Najčešći definisanja pomoću Web Ontology Language (OWL) standarda
				\begin{itemize}
					\item OWL Full
					\item OWL DL
					\item OWL Lite
					\item Može se enkodirati u XML/RDF format
				\end{itemize}
			\item Razni alati
				\begin{itemize}
					\item Protégé
					\item ClueGO
				\end{itemize}
		\end{itemize}
	\end{center}
  \end{frame}  
  \begin{frame}
    \frametitle{Metodi}
    	\framesubtitle{Gene Ontology}	
		\begin{center}		
		\begin{itemize}
			\item Nastala iz saradnje baza genomskih podataka
				\begin{itemize}
					\item FlyBase
					\item Mouse Genome Informatics
					\item Saccharomyces Genome Database
				\end{itemize}
			\item Mnoge DNK sekvence kao i njihove funkcije očuvane medu različitim vrstama
			\item Biomedical Ontologies Foundry
			\item Tri odvojene ontologije
				\begin{itemize}
					\item Bioloških procesa
					\item Molekularnih funkcija
					\item Ćelijskih komponenti
				\end{itemize}
			\item AmiGO
			\item Go.db, RamiGO
			\item Analiza zastupljenosti
		\end{itemize}
	\end{center}
  \end{frame}  
  \begin{frame}
    \frametitle{Metodi}
    	\framesubtitle{Biological Pathways Exchange}	
		\begin{center}		
		\begin{itemize}
			\item Modelira biološke (metaboličke, genetske, signalne) puteve
			\item Klase
				\begin{itemize}
					\item fizički entiteti
					\item interakcije
					\item putevi
				\end{itemize}
			\item Veliki broj različitih baza
				\begin{itemize}
					\item Kyoto Encyclopedia of Genes and Genomes
					\item Reactome
					\item WikiPathways
					\item Pathway Interaction Database
					\item \dots
				\end{itemize}
			\item Takođe postoje alati za R
				\begin{itemize}
					\item PaxTools
					\item rBioxParser
				\end{itemize}
		\end{itemize}
	\end{center}
  \end{frame}  
  \begin{frame}
    \frametitle{Zaključak}
		\begin{center}		
		\begin{itemize}
			\item Neophodne za koordinaciju više izvora podataka
			\item Opšteprihvaćene u biomedicini
			\item Moćan način enkodiranja znanja
			\item Omogućavaju nove pristupe problemima
		\end{itemize}
	\end{center}
  \end{frame} 
\end{document}