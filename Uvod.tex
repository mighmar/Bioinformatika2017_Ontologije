\section{Uvod}
\label{sec:uvod}

U poslednjim decenijama prikupljena je velika količina podataka iz biomedicine.
Da bi rad sa ovim podacim bio moguć ono moraju biti skladišteni na način koji je
dobro dokumentovan i omogućava lak pristup.
Nastanak ontologija u računarstvu pružio je odgovarajući način za modelovanje specifičnih znanja.
Ontologije je se danas često koriste za rad sa biološkim i medicinskim znanjem.

\subsection{Teorijska pozadina}

	Naziv "ontologija"\  potiče iz filosofije, gde se odnosi na proučavanje postojanja i realnosti
kao grana metafizike koju je osnovao Aristotel.
Definicija ontologije u informatici:
\begin{displayquote}
Specifikacija reprezentativnog vokabulara za deljeni domen diskursa
-definicije klasa, relacija, funkcija i drugih obejkata-
zove se ontologija.
\end{displayquote}

Ontologije se svode na apstrakciju, tj. na uprošćeni pogled domena koji se modeluje.
One definisu klase objekate i relacije među njima. Svojstva definišu atribute i osobine klasa.
\par

Glavni ciljevi kreiranja ontologija su formalizacija strukture specifičnih informacija o domenu,
razdvajanje znanja od strukturi podataka od samih podataka i omogućivanje ponovnog korišćenja 
i deljenje strukutre i znanja.
Moguće je modelovati opisne logike koje omogućuju automatsko rezonovanje i zaključivanje
na osnovu baze znanja i logičkih operacija.
\par

U praksi ontologije se najčešće koriste kada je potrebno dodati sloj apstrakcije
kada je domen kompleksan i postoji velika razlika u granularnosti znanja.
Na primer, rečenica: "Gen A aktivira gen B" i naučni rad koji zaključuje da gen A
aktivira gen B na visokom nivou apstrakcije su jednaki. Do ovog zaključka se ne može
doći poređenjem teksta naučnog rada i rečenice.
