\section{Materijali}

U poslednjoj deceniji predložen, definisan i objavljen je veliki broj ontologija iz biomedicine.
Neke od najznačajnijih su Chemical Entities of Biological Interest (ChEBI), Gene Ontology (GO), 
i ontologiju za Biological Pathways Exchange (BioPAX). \par
 ChEBI sadrži informacije o malim molekulima i molekularnim entitetima 
koji se javljaju u metaboličkim procesima, kao laboratorjiski reagensi, farmaceutski lekovi i subatmske čestice.
 Kompleksni markomolekuli, poput proteina, obično nisu uključeni. Ideja je da ChEBI bude opširan rečnik biohemijskih entiteta, njihovih
za mašine čitljivih strukturnih informacija i primena.

\subsection{Pronalaženje i pristup ontologijama}

Internet strane omogućavaju pronalaženje i pristup ontologijama.\\*
ChEBI i GO su deo Biomedical Ontologies Foundry\footnote{OBO,\url{www.obofoundry.org}}, kolektiva za razvoj ontologija čiji je cilj standardizacija razvoja biomedicinskih ontologija i njihovo povezivanje. 
Takođe OBO sadrži veliki broj prihvaćenih ontologija i ontologija kandidata.
BioPortal\footnote{ \url{http://bioportal.bioontology.org/}} i EMBL-EBI Ontology Lookup Service\footnote{ \url{www.ebi.ac.uk/ontology-lookup/}}
su su neke od internet strana koje omogućavaju pretraživanje ontologija.

\subsection{Enkodiranje ontologija}

Postoji veli broj alata i softverskih rešenja koja omogućavaju deinisanje novih i manjanje postojećih onotologija, kao i 
enkodiranje i modifkovanje podataka u skladu sa postojećom definicijom ontologije.
Najčešći način za definisanje ontologija je pomoću Web Ontology Language (OWL), standarda World Wide Web Consortium (W3C).
Ovako definisana ontologija može biti enkodirana u XML/RDF format. RDF je format koji definise trojke oblika subjekat-predikat-objekat
za formiranje izraza. \par
Postoje 3 verzije OWL: OWL Full, OWL DL i OWL Lite.
OWL Lite pokriva osnove modeliranja podataka i ova verzija je odogvarajuća u većini slučajeva.
OWL DL je nadogradnja OWL Lite koja uvodi opisnu logiku i mogućnost zaključivanja. Ova verzija je nophodna za logičko zaključivanje.
OWL Full podržava kompletnu OWL sintaksu i omogćava proširivanje i nadogradnju postojećeg OWL rečnika.

\subsection{Editovanje ontologija}

Prot\'eg\'e je veoma popularan softverski alat koji omogućava editovanje ontologija kao i unos podataka u ontologije.
Alat se koristi i u akademiji i u industriji.
Otoverenog je koda i dostupan na svim platformama.
Takođe postoji i internet verzija alata, koja može biti preuzeta i postavljena na lokalni server 
(ova verzija je dostupa svima na internet strani Stanforda webprotege.stanford.edu). \par
Dostupni su i razni alata za specifične ontologije. ClueGO je dodatak za Cytoscape(alat za vizualizaciju)
 koji omogućava vizualizaciju kompleksnih odnosa u ontologiji gena (GO).