\section{Metodi}
\label{sec:metodi}

U bioinformatici, najčešće korišćene ontologije su Gene Ontology (GO) i Biological Pathways Exchange Ontology (BioPAX).

\subsection{Gene Ontology}

GO ili \emph{Ontologija gena} nastala je iz saradnje tri baze genomskih podataka, FlyBase, Mouse Genome Informatics i Saccharomyces Genome Database koje skladište podatke o genomima voćnih mušica, laboratorijskih miševa i pekarskog kvasca redom.
Ovakva saradnja se razvila jer se pokazalo da su mnoge DNK sekvence kao i njihova funkcija očuvane među različitim vrstama, te je nastala potreba za usaglašenjem rečnika kojim se opisuju funkcije gena opisanih u ovim bazama. Projekat je veoma živ i ažurira se na dnevnoj bazi, njegov staralac je GO Konzorcijum, udruženje od oko 30 istraživačkih grupa.

Tri ontologije od kojih se GO sastoji su ontologija bioloških procesa, molekularnih funkcija i ćelijskih komponenti. Biološki procesi su svi događaji i tokovi unutar ćelije i organizma, molekularne funkcije podrazumevaju dejstva proteina u ćeliji a ćelijske komponente obuhvataju delove (eukariotske) ćelije i njena mikrookruženja.

Uprkos nazivu, GO ne obuhvata biološke objekte poput gena i proteina koje proizvode već samo njihova svojstva odnosno pojave u kojima učestvuju. 

\subsubsection{Pristup}
Podacima se može pristupiti pomoću zvaničnog veb interfejsa AmiGO\footnote{\url{http://amigo.geneontology.org/amigo/}} koji omogućava više vrsta pretraga, kao i direktno pregledanje hijerarhije klasa.

Druga mogućnost je skidanje potpune ontologije\footnote{\url{http://www.geneontology.org/page/download-ontology}} za upotrebu pomoću alata poput Protégé-a .
Pritom je dostupno više verzija ontologije u zavisnosti od potreba korisnika, poput izdvojenih podskupova namenjenih za rad na pojedinačnim genomima kao i filtriranih verzija čije klase ne stoje u cikličnim odnosima koje se koriste pri stvaranju novih anotacija.

Postoje takođe i paketi za R, poput GO.db koji operiše sa GO hijerarhijom klasa i RamiGO koji funkcioniše kao interfejs ka AmiGO servisu.

\subsubsection{Upotreba}

Zahvaljujući sada već ogromnom broju podataka koji su kompatibilni sa GO-em, razvili su se novi metodi u bioinformatici koji koriste GO taksonomiju termina za rešavanje bioloških problema. 

Najuspešnija takva metoda je \emph{analiza zastupljenosti} (eng. enrichment analysis), koja u listama gena ekspresovanim u različitim uslovima sastavljenih putem visokopropusnih eksperimenata traže prezastupljenost odnosno podzastupljenost određenih termina iz ontologije i utvrđuje da li su statistički značajni.

Postoji više implementacija analize zastupljenosti za GO, među njima su ona dostupna pomoću AmiGO serivisa, GOstat\footnote{\url{gostat.wehi.edu.au}} i DAVID\footnote{\url{david.abcc.ncifcrf.gov}} sajtova. Kao i pomoću Cytoscape dodataka clueGO i BiNGO i R paketa topGO.




\subsection{Biological Pathways Exchange}

Biological Pathways Exchange ili BioPAX je ontologija zasnovana na RDF/OWL formatu koja se bavi standardizacijom termina koji se tiču bioloških puteva. 
Pored nje, istu namenu imaju i Systems Biology Markup Language (SBML) zasnovan na XML-u koji je proizvode iste inicijative (Computational Modeling in Biology Network ili COMBINE) i Human Proteome Organizations Proteomics Standards Initiative’s Molecular Interaction format (PSI-MI).

Klase koje čine srž BioPAX-a su fizički entiteti (molekuli, DNK, RNK, proteini, kompleksi), interakcije (hemijske reakcije) i biološki putevi (nizovi interakcija). Trenutno ontologija sadrži oko 70 klasa.

Postoji veliki broj baza podataka koje se odnose na biološke puteve, \url{pathguide.org} navodi čak 569 resursa vezanih za ovu temu, od kojih oko 80 zadovoljava BioPAX standard.
Najistaknutije od ovih su Kyoto Encyclopedia of Genes and Genomes (KEGG), Reactome, WikiPathways i Pathway Interaction Database.

Iako se BioPAX enkodiranim podacima može pristupiti i pomoću opšteg programa poput Protégé-a, postoje takođe i alati namenjeni za BioPAX specifično, poput Paxtools-a i rBiopaxParser paketa za R.
Ovaj paket sadrži nekoliko funkcija koje olakšavaju rukovanje BioPAX modelom, između ostalog, manipulisanje podacima, čuvanje izmenjenih ontologija kao i pretvaranje podataka u grafove interakcije koji se dalje mogu koristiti u algoritmima za rekonstruisanje bioloških mreža.






