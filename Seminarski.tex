% !TEX encoding = UTF-8 Unicode

\documentclass[a4paper]{article}

\usepackage{color}
\usepackage{url}
\usepackage[T2A]{fontenc} % enable Cyrillic fonts
\usepackage[utf8]{inputenc} % make weird characters work
\usepackage{graphicx}
\usepackage{csquotes}
\usepackage[english,serbian]{babel}
%\usepackage[english,serbianc]{babel} %ukljuciti babel sa ovim opcijama, umesto gornjim, ukoliko se koristi cirilica

\usepackage[unicode]{hyperref}
\hypersetup{colorlinks,citecolor=green,filecolor=green,linkcolor=blue,urlcolor=blue}


\begin{document}

\title{Ontologije \\ \small{Seminarski rad u okviru kursa\\Bioinformatika\\ Matematički fakultet}}

\author{Marko Crnobrnja, Marko Mićić}
\date{12.~maj 2017.}
\maketitle

\abstract{Ontologije su popularan i moćan način za enkodiranje podataka u odgovarajući format i upravljenje znanjem u nekom domenu. 
U ovom poglavlju date su ukratko teorijske osnove ontologija i prikazane dve bitne ontologije iz oblasti biomedicine: The Gene Ontology i 
ontologija za Biological Pathways Exchange. Takođe opisani su i neki alati za rad sa ontologijama. }

\tableofcontents

\newpage

\section{Uvod}
\label{sec:uvod}

U poslednjim decenijama prikupljena je velika količina podataka iz biomedicine.
Da bi rad sa ovim podacim bio moguć ono moraju biti skladišteni na način koji je
dobro dokumentovan i omogućava lak pristup.
Nastanak ontologija u računarstvu pružio je odgovarajući način za modelovanje specifičnih znanja.
Ontologije je se danas često koriste za rad sa biološkim i medicinskim znanjem.

\subsection{Teorijska pozadina}

	Naziv "ontologija"\  potiče iz filosofije, gde se odnosi na proučavanje postojanja i realnosti
kao grana metafizike koju je osnovao Aristotel.
Definicija ontologije u informatici:
\begin{displayquote}
Specifikacija reprezentativnog vokabulara za deljeni domen diskursa
-definicije klasa, relacija, funkcija i drugih obejkata-
zove se ontologija.
\end{displayquote}

Ontologije se svode na apstrakciju, tj. na uprošćeni pogled domena koji se modeluje.
One definisu klase objekate i relacije među njima. Svojstva definišu atribute i osobine klasa.
\par

Glavni ciljevi kreiranja ontologija su formalizacija strukture specifičnih informacija o domenu,
razdvajanje znanja od strukturi podataka od samih podataka i omogućivanje ponovnog korišćenja 
i deljenje strukutre i znanja.
Moguće je modelovati opisne logike koje omogućuju automatsko rezonovanje i zaključivanje
na osnovu baze znanja i logičkih operacija.
\par

U praksi ontologije se najčešće koriste kada je potrebno dodati sloj apstrakcije
kada je domen kompleksan i postoji velika razlika u granularnosti znanja.
Na primer, rečenica: "Gen A aktivira gen B" i naučni rad koji zaključuje da gen A
aktivira gen B na visokom nivou apstrakcije su jednaki. Do ovog zaključka se ne može
doći poređenjem teksta naučnog rada i rečenice.
 

\section{Materijali}

U poslednjoj deceniji predložen, definisan i objavljen je veliki broj ontologija iz biomedicine.
Neke od najznačajnijih su Chemical Entities of Biological Interest (ChEBI), Gene Ontology (GO), 
i ontologiju za Biological Pathways Exchange (BioPAX). \par
 ChEBI sadrži informacije o malim molekulima i molekularnim entitetima 
koji se javljaju u metaboličkim procesima, kao laboratorjiski reagensi, farmaceutski lekovi i subatmske čestice.
 Kompleksni markomolekuli, poput proteina, obično nisu uključeni. Ideja je da ChEBI bude opširan rečnik biohemijskih entiteta, njihovih
za mašine čitljivih strukturnih informacija i primena.

\subsection{Pronalaženje i pristup ontologijama}

Internet strane omogućavaju pronalaženje i pristup ontologijama.\\*
ChEBI i GO su deo Biomedical Ontologies Foundry\footnote{OBO,\url{www.obofoundry.org}}, kolektiva za razvoj ontologija čiji je cilj standardizacija razvoja biomedicinskih ontologija i njihovo povezivanje. 
Takođe OBO sadrži veliki broj prihvaćenih ontologija i ontologija kandidata.
BioPortal\footnote{ \url{http://bioportal.bioontology.org/}} i EMBL-EBI Ontology Lookup Service\footnote{ \url{www.ebi.ac.uk/ontology-lookup/}}
su su neke od internet strana koje omogućavaju pretraživanje ontologija.

\subsection{Enkodiranje ontologija}

Postoji veli broj alata i softverskih rešenja koja omogućavaju deinisanje novih i manjanje postojećih onotologija, kao i 
enkodiranje i modifkovanje podataka u skladu sa postojećom definicijom ontologije.
Najčešći način za definisanje ontologija je pomoću Web Ontology Language (OWL), standarda World Wide Web Consortium (W3C).
Ovako definisana ontologija može biti enkodirana u XML/RDF format. RDF je format koji definise trojke oblika subjekat-predikat-objekat
za formiranje izraza. \par
Postoje 3 verzije OWL: OWL Full, OWL DL i OWL Lite.
OWL Lite pokriva osnove modeliranja podataka i ova verzija je odogvarajuća u većini slučajeva.
OWL DL je nadogradnja OWL Lite koja uvodi opisnu logiku i mogućnost zaključivanja. Ova verzija je nophodna za logičko zaključivanje.
OWL Full podržava kompletnu OWL sintaksu i omogćava proširivanje i nadogradnju postojećeg OWL rečnika.

\subsection{Editovanje ontologija}

Prot\'eg\'e je veoma popularan softverski alat koji omogućava editovanje ontologija kao i unos podataka u ontologije.
Alat se koristi i u akademiji i u industriji.
Otoverenog je koda i dostupan na svim platformama.
Takođe postoji i internet verzija alata, koja može biti preuzeta i postavljena na lokalni server 
(ova verzija je dostupa svima na internet strani Stanforda webprotege.stanford.edu). \par
Dostupni su i razni alata za specifične ontologije. ClueGO je dodatak za Cytoscape(alat za vizualizaciju)
 koji omogućava vizualizaciju kompleksnih odnosa u ontologiji gena (GO).

\section{Metodi}
\label{sec:metodi}

U bioinformatici, najčešće korišćene ontologije su Gene Ontology (GO) i Biological Pathways Exchange Ontology (BioPAX).

\subsection{Gene Ontology}

GO ili \emph{Ontologija gena} nastala je iz saradnje tri baze genomskih podataka, FlyBase, Mouse Genome Informatics i Saccharomyces Genome Database koje skladište podatke o genomima voćnih mušica, laboratorijskih miševa i pekarskog kvasca redom.
Ovakva saradnja se razvila jer se pokazalo da su mnoge DNK sekvence kao i njihova funkcija očuvane među različitim vrstama, te je nastala potreba za usaglašenjem rečnika kojim se opisuju funkcije gena opisanih u ovim bazama. Projekat je veoma živ i ažurira se na dnevnoj bazi, njegov staralac je GO Konzorcijum, udruženje od oko 30 istraživačkih grupa.

Tri ontologije od kojih se GO sastoji su ontologija bioloških procesa, molekularnih funkcija i ćelijskih komponenti. Biološki procesi su svi doga\-đaji i tokovi unutar ćelije i organizma, molekularne funkcije podrazumevaju dejstva proteina u ćeliji a ćelijske komponente obuhvataju delove (eukariotske) ćelije i njena mikrookruženja.

Uprkos nazivu, GO ne obuhvata biološke objekte poput gena i proteina koje proizvode već samo njihova svojstva odnosno pojave u kojima učestvuju. 

\subsubsection{Pristup}
Podacima se može pristupiti pomoću zvaničnog veb interfejsa AmiGO\footnote{\url{http://amigo.geneontology.org/amigo/}} koji omogućava više vrsta pretraga, kao i direktno pregledanje hijerarhije klasa.

Druga mogućnost je skidanje potpune ontologije\footnote{\url{http://www.geneontology.org/page/download-ontology}} za upotrebu pomoću alata poput Protégé-a .
Pritom je dostupno više verzija ontologije u zavisnosti od potreba korisnika, poput izdvojenih podskupova namenjenih za rad na pojedinačnim genomima kao i filtriranih verzija čije klase ne stoje u cikličnim odnosima koje se koriste pri stvaranju novih anotacija.

Postoje takođe i paketi za R, poput GO.db koji operiše sa GO hijerarhijom klasa i RamiGO koji funkcioniše kao interfejs ka AmiGO servisu.

\subsubsection{Upotreba}

Zahvaljujući sada već ogromnom broju podataka koji su kompatibilni sa GO-em, razvili su se novi metodi u bioinformatici koji koriste GO taksonomiju termina za rešavanje bioloških problema. 

Najuspešnija takva metoda je \emph{analiza zastupljenosti} (eng. enrichment analysis), koja u listama gena ekspresovanim u različitim uslovima sastavljenih putem visokopropusnih eksperimenata traže prezastupljenost odnosno podzastupljenost određenih termina iz ontologije i utvrđuje da li su statistički značajni.

Postoji više implementacija analize zastupljenosti za GO, među njima su ona dostupna pomoću AmiGO serivisa, GOstat\footnote{\url{gostat.wehi.edu.au}} i DAVID\footnote{\url{david.abcc.ncifcrf.gov}} sajtova. Kao i pomoću Cytoscape dodataka clueGO i BiNGO i R paketa topGO.




\subsection{Biological Pathways Exchange}

Biological Pathways Exchange ili BioPAX je ontologija zasnovana na RDF/OWL formatu koja se bavi standardizacijom termina koji se tiču bioloških puteva. 
Pored nje, istu namenu imaju i Systems Biology Markup Language (SBML) zasnovan na XML-u koji je proizvode iste inicijative (Computational Modeling in Biology Network ili COMBINE) i Human Proteome Organizations Proteomics Standards Initiative’s Molecular Interaction format (PSI-MI).

Klase koje čine srž BioPAX-a su fizički entiteti (molekuli, DNK, RNK, proteini, kompleksi), interakcije (hemijske reakcije) i biološki putevi (nizovi interakcija). Trenutno ontologija sadrži oko 70 klasa.

Postoji veliki broj baza podataka koje se odnose na biološke puteve, \url{pathguide.org} navodi čak 569 resursa vezanih za ovu temu, od kojih oko 80 zadovoljava BioPAX standard.
Najistaknutije od ovih su Kyoto Encyclopedia of Genes and Genomes (KEGG), Reactome, WikiPathways i Pathway Interaction Database.

Iako se BioPAX enkodiranim podacima može pristupiti i pomoću op\-šteg programa poput Protégé-a, postoje takođe i alati namenjeni za BioPAX specifično, poput Paxtools-a i rBiopaxParser paketa za R.

Ovaj paket sadrži nekoliko funkcija koje olakšavaju rukovanje BioPAX modelom, između ostalog, manipulisanje podacima, čuvanje izmenjenih ontologija kao i pretvaranje podataka u grafove interakcije koji se dalje mogu koristiti u algoritmima za rekonstruisanje bioloških mreža.








\end{document}
 
