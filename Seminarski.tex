% !TEX encoding = UTF-8 Unicode

\documentclass[a4paper]{article}

\usepackage{color}
\usepackage{url}
\usepackage[T2A]{fontenc} % enable Cyrillic fonts
\usepackage[utf8]{inputenc} % make weird characters work
\usepackage{graphicx}
\usepackage{csquotes}
\usepackage[english,serbian]{babel}
%\usepackage[english,serbianc]{babel} %ukljuciti babel sa ovim opcijama, umesto gornjim, ukoliko se koristi cirilica

\usepackage[unicode]{hyperref}
\hypersetup{colorlinks,citecolor=green,filecolor=green,linkcolor=blue,urlcolor=blue}


\begin{document}

\title{Ontologije \\ \small{Seminarski rad u okviru kursa\\Bioinformatika\\ Matematički fakultet}}

\author{Marko Crnobrnja, Marko Mićić}
\date{12.~maj 2017.}
\maketitle

\abstract{Ontologije su popularan i moćan način za enkodiranje podataka u odgovarajući format i upravljenje znanjem u nekom domenu. 
U ovom poglavlju date su ukratko teorijske osnove ontologija i prikazane dve bitne ontologije iz oblasti biomedicine: The Gene Ontology i 
ontologija za Biological Pathways Exchange. Takođe opisani su i neki alati za rad sa ontologijama. }

\tableofcontents

\newpage

\section{Uvod}
\label{sec:uvod}

U poslednjim decenijama prikupljena je velika količina podataka iz biomedicine.
Da bi rad sa ovim podacim bio moguć ono moraju biti skladišteni na način koji je
dobro dokumentovan i omogućava lak pristup.
Nastanak ontologija u računarstvu pružio je odgovarajući način za modelovanje specifičnih znanja.
Ontologije je se danas često koriste za rad sa biološkim i medicinskim znanjem.

\subsection{Teorijska pozadina}

	Naziv "ontologija"\  potiče iz filosofije, gde se odnosi na proučavanje postojanja i realnosti
kao grana metafizike koju je osnovao Aristotel.
Definicija ontologije u informatici:
\begin{displayquote}
Specifikacija reprezentativnog vokabulara za deljeni domen diskursa
-definicije klasa, relacija, funkcija i drugih obejkata-
zove se ontologija.
\end{displayquote}

Ontologije se svode na apstrakciju, tj. na uprošćeni pogled domena koji se modeluje.
One definisu klase objekate i relacije među njima. Svojstva definišu atribute i osobine klasa.
\par

Glavni ciljevi kreiranja ontologija su formalizacija strukture specifičnih informacija o domenu,
razdvajanje znanja od strukturi podataka od samih podataka i omogućivanje ponovnog korišćenja 
i deljenje strukutre i znanja.
Moguće je modelovati opisne logike koje omogućuju automatsko rezonovanje i zaključivanje
na osnovu baze znanja i logičkih operacija.
\par

U praksi ontologije se najčešće koriste kada je potrebno dodati sloj apstrakcije
kada je domen kompleksan i postoji velika razlika u granularnosti znanja.
Na primer, rečenica: "Gen A aktivira gen B" i naučni rad koji zaključuje da gen A
aktivira gen B na visokom nivou apstrakcije su jednaki. Do ovog zaključka se ne može
doći poređenjem teksta naučnog rada i rečenice.
 

\section{Materijali}

U poslednjoj deceniji predložen, definisan i objavljen je veliki broj ontologija iz biomedicine.
Neke od najznačajnijih su Chemical Entities of Biological Interest (ChEBI), Gene Ontology (GO), 
i ontologiju za Biological Pathways Exchange (BioPAX). \par
 ChEBI sadrži informacije o malim molekulima i molekularnim entitetima 
koji se javljaju u metaboličkim procesima, kao laboratorjiski reagensi, farmaceutski lekovi i subatomske čestice.
 Kompleksni markomolekuli, poput proteina, obično nisu uključeni. Ideja je da ChEBI bude opširan rečnik biohemijskih entiteta, njihovih
za mašine čitljivih strukturnih informacija i primena.

\subsection{Pronalaženje i pristup ontologijama}

Internet strane omogućavaju pronalaženje i pristup ontologijama.\\*
ChEBI i GO su deo Biomedical Ontologies Foundry\footnote{OBO,\url{www.obofoundry.org}}, kolektiva za razvoj ontologija čiji je cilj standardizacija razvoja biomedicinskih ontologija i njihovo povezivanje. 
Takođe OBO sadrži veliki broj prihvaćenih ontologija i ontologija kandidata.
BioPortal\footnote{ \url{http://bioportal.bioontology.org/}} i EMBL-EBI Ontology Lookup Service\footnote{ \url{www.ebi.ac.uk/ontology-lookup/}}
su su neke od internet strana koje omogućavaju pretraživanje ontologija.

\subsection{Enkodiranje ontologija}

Postoji veli broj alata i softverskih rešenja koja omogućavaju deinisanje novih i manjanje postojećih onotologija, kao i 
enkodiranje i modifkovanje podataka u skladu sa postojećom definicijom ontologije.
Najčešći način za definisanje ontologija je pomoću Web Ontology Language (OWL), standarda World Wide Web Consortium (W3C).
Ovako definisana ontologija može biti enkodirana u XML/RDF format. RDF je format koji definise trojke oblika subjekat-predikat-objekat
za formiranje izraza. \par
Postoje 3 verzije OWL: OWL Full, OWL DL i OWL Lite.
OWL Lite pokriva osnove modeliranja podataka i ova verzija je odogvarajuća u većini slučajeva.
OWL DL je nadogradnja OWL Lite koja uvodi opisnu logiku i mogućnost zaključivanja. Ova verzija je nophodna za logičko zaključivanje.
OWL Full podržava kompletnu OWL sintaksu i omogćava proširivanje i nadogradnju postojećeg OWL rečnika.

\subsection{Editovanje ontologija}

Prot\'eg\'e je veoma popularan softverski alat koji omogućava editovanje ontologija kao i unos podataka u ontologije.
Alat se koristi i u akademiji i u industriji.
Otoverenog je koda i dostupan na svim platformama.
Takođe postoji i internet verzija alata, koja može biti preuzeta i postavljena na lokalni server 
(ova verzija je dostupa svima na internet strani Stanforda\footnote{\url{http://webprotege.stanford.edu}}). \par
Dostupni su i razni alata za specifične ontologije. ClueGO je dodatak za Cytoscape(alat za vizualizaciju)
 koji omogućava vizualizaciju kompleksnih odnosa u ontologiji gena (GO).


\section{Metodi}
\label{sec:metodi}

\end{document}
 
