% !TEX encoding = UTF-8 Unicode

\documentclass[a4paper]{article}

\usepackage{color}
\usepackage{url}
\usepackage[T2A]{fontenc} % enable Cyrillic fonts
\usepackage[utf8]{inputenc} % make weird characters work
\usepackage{graphicx}

\usepackage[english,serbian]{babel}
%\usepackage[english,serbianc]{babel} %ukljuciti babel sa ovim opcijama, umesto gornjim, ukoliko se koristi cirilica

\usepackage[unicode]{hyperref}
\hypersetup{colorlinks,citecolor=green,filecolor=green,linkcolor=blue,urlcolor=blue}


\begin{document}

\title{Ontologije \\ \small{Seminarski rad u okviru kursa\\Bioinformatika\\ Matematički fakultet}}

\author{Marko Crnobrnja, Marko Mićić}
\date{12.~maj 2017.}
\maketitle

\abstract{}


\newpage
\tableofcontents

\newpage

\section{Uvod}
\label{sec:uvod}

\section{Materijali}

U poslednjoj deceniji predložen, definisan i objavljen je veliki broj ontologija iz biomedicine.
Neke od najznačajnijih su Chemical Entities of Biological Interest (ChEBI), Gene Ontology (GO), 
i ontologiju za Biological Pathways Exchange (BioPAX). ChEBI sadrži informacije o malim molekulima i molekularnim entitetima 
koji se javljaju u metaboličkim procesima, kao laboratorjiski reagensi, farmaceutski lekovi i subatmske čestice.
 Kompleksni markomolekuli, poput proteina, obično nisu uključeni. Ideja je da ChEBI bude opširan rečnik biohemijskih entiteta, njihovih
za mašine čitljivih strukturnih informacija i primena.

\subsection{Pronalaženje i pristup ontologijama}




\end{document}
 
